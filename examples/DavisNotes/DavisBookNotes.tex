\documentclass[11pt]{amsart}
\usepackage[utf8]{inputenc}
\usepackage{graphicx,color,epsfig}
\usepackage{amsmath,amsthm,amsfonts,amssymb}
\usepackage{epstopdf}
%\usepackage{hyperref}
\usepackage{bbm,dsfont} % used for \one
\usepackage{fullpage}


\newtheorem{theorem}{Theorem}[section]
\newtheorem{corollary}[theorem]{Corollary}
\newtheorem{lemma}[theorem]{Lemma}
\newtheorem{proposition}[theorem]{Proposition}
\newtheorem*{conj}{Conjecture}
\newtheorem*{theorem*}{Theorem}

\theoremstyle{definition}
\newtheorem{definition}[theorem]{Definition}
\newtheorem*{terminology}{Terminology}
\newtheorem*{notation}{Notation}

\theoremstyle{remark}
\newtheorem{example}[theorem]{Example}
\newtheorem{remark}[theorem]{Remark}


\newcommand{\rank}[1]{}
\newcommand{\depends}[1]{}
\newcommand{\summary}[1]{}
\newcommand{\mainText}[1]{#1}

\newcommand{\extgen}{\bar{A}}
\newcommand{\Feqgen}{\mathcal{F}_{\extgen}}
\newcommand{\VecField}{\Xi}


\begin{document}

\section{General results}\label{sec:general-res}

\begin{definition}[local maringale]\label{def:loc-martingale}
TODO
\end{definition}

\begin{definition}[bounded variation]\label{def:bounded-variation}
TODO
\end{definition}

\begin{proposition}\label{prop:loc-martingale-finite-variation-cst}
A continuous local martingale with finite variations is almost surely constant i.e.
\[
	P(M_t = 0 \, \forall t) = 1
\]
\end{proposition}
\begin{proof}
th6.13 at 
http://www.stats.ox.ac.uk/~etheridg/ctsmartingales.pdf
\end{proof}

\section{Extended generator of a PDMP}\label{sec:Ext-gen-Markov}
\depends{}

\begin{definition}[Domain of the extended generator of a Markov Process]\label{def:ext-gen-domain}
\depends{def:loc-martingale}
\summary{TEST SUMMARY}
\mainText{Let $\mathcal{D}(\extgen)$ be the subset of measurable functions $f:E\rightarrow \mathbb{R}$ such that there 
exists a measurable function $h: E \rightarrow \mathbb{R}$ satisfying for every $x_0\in E$:
\begin{itemize}
	\item For $P_{x_0}$ almost every $\omega$, $t\rightarrow f(x_t(\omega))$ is integrable
	\item The process defined by
	\[
		C_t^{f,h} = f(x_t) - f(x_0) - \int_0^t h(x_s) ds
	\]
	is a local martingale (with respect to $P_{x_0}$).
\end{itemize}}
\end{definition}

\begin{definition}[Potential zero sets]\label{def:potential-zero-set}
\begin{enumerate}
\item Let $A \in \mathcal{E}$ where $\mathcal{E}$ is the sigma algebra of $E$. We say that $A$ has zero potential if for every $x_0\in E$, 
\[
	\int_0^\infty 1_A(x_s) ds = 0 \qquad P_{x_0} \text{ almost surely}.
\]
For those sets, the process 'spends no time' in $A$ for any starting point in $E$. 
\item We call $\Feqgen$ the equivalent classes of functions that are equal everywhere but on a potential zero set.
\end{enumerate}
\end{definition}

\begin{proposition}\label{prop:eq-class-ext-gen}
\depends{def:ext-gen-domain,def:potential-zero-set}
Let $f \in \mathcal{D}(\extgen)$.
\begin{enumerate}
	\item Let $h_1,h_2$ measurable functions from $E$ to $\mathbb{R}$ such that $C_t^{f,h_1}$ and $C_t^{f,h_2}$ are local martingales. Then the set 
	\[ 
	 \{x \in E \, | \,  h_1(x) \neq h_2(x) \}
	\]
	is of potential zero, i.e. there exists $h\in \Feqgen$ such that $h_1$ and $h_2$ are representations of $h$.
	\item  Let $h_1$ measurable functions from $E$ to $\mathbb{R}$ such that $C_t^{f,h_1}$. Let $h_2$ be measurable function $h_2: E \rightarrow \mathbb{R}$ such that $h_2 = h_1$ outside of a zero potential set. Then $C_t^{f,h_2}$ is a local martingale (and $C_t^{f,h_1}  = C_t^{f,h_2}$).
\end{enumerate}
\end{proposition}
\begin{proof}
1/ Let $A = \{x \in E \, | \, h_1(x) \neq h_2(x) \}$
We have that for every $x_0 \in E$,
\[
	M_t = C_t^{f,h_1} - C_t^{f,h_2} = \int_0^t h_1(x_s) - h_2(x_s) ds
\]
is a local martingale. Furthermore, due to its integral form, it is continuous and of bounded variation. Since $M_0 = $ and using prop \ref{prop:loc-martingale-finite-variation-cst}, we see that $P_{x_0}$ almost surely, $M_t = 0$ for all $t \geq 0$.
For a fixed $\omega$ such that $M_t = 0$ for all $t$, $t \mapsto M_t(\omega)$ admits $h_1(x_t(\omega)) - h_2(x_t(\omega))$ as a weak derivative. This implies that $t \mapsto M_t(\omega)$ has derivative for almost every $t$ and $M_t(\omega)' = h_1(x_t(\omega)) - h_2(x_t(\omega))$ for almost every $t$. Since $M_t(\omega)' = 0$, this implies that $h_1(x_t(\omega)) - h_2(x_t(\omega)) = 0$ for almost every $t$, which implies that
\[
	\int_0^\infty 1_A(x_s(\omega)) ds = 0.
\]
Hence $A$ is a zero potential set.

2/ It is easy to see that $C_t^{f,h_1}  = C_t^{f,h_2}$.
\end{proof}

\begin{definition}[Extended generator]\label{def:ext-gen}
\depends{prop:eq-class-ext-gen,def:potential-zero-set,def:ext-gen-domain}
For any $f \in \mathcal{D}(\extgen)$, we write $\extgen f \in \Feqgen$ the equivalent class of function such that for every representation $h$ of $\extgen f$, $C_t^{f,h}$ is a local martingale.
\end{definition}

\begin{proposition}[Link of infinitesimal generator with extended generator]\label{prop:link-gen-ext-gen}
\depends{def:ext-gen}
\begin{enumerate}
	\item $\mathcal{D}(A) \subset \mathcal{D}(\extgen)$
	\item $Af = \extgen f$ for all $f \in \mathcal{D}(A)$ (in the sense that $Af$ is a function of the equivalent class of functions $\extgen f$)
\end{enumerate}
\end{proposition}
\begin{proof}
If $f \in \mathcal{D}(A)$, then $C_t^{f,Af}$ is a Martingale (Dynkin formula), hence it is a local martingale, which implies the result.
\end{proof}

\section{Construction of a PDMP}\label{sec:constr-pdmp}

\begin{definition}\label{def:point-process-p}
Let $(x_t)$ be a jump process  taking values in $X \cup \{\Delta_\infty\}$ with $X$ a Borel space and $\Delta_\infty$ an isolated point. For $A \in \mathcal{B}(X)$ (the Borel sets of $X$?) and $t \geq $ we define
\[
	p(t,A): \omega \mapsto 1_{t \geq T}(\omega) 1_{Z\in A}(\omega)
\]
\end{definition}

\begin{definition}[$L_1(p)$]\label{def:L1-p}

\end{definition}

\begin{definition}[$L_1^{loc}(p)$]\label{def:L1-loc-p}
We say that $g \in L_1^{loc}(p)$ if $g \, 1_{t < \tau_n} \in L_1(p)$. WHAT IS $\tau_n$?
\end{definition}


\section{Extended generator of a PDMP}\label{sec:ext-gen-pdmp}

\begin{theorem}[Extended generator of a PDMP]\label{th:ext-gen-pdmp}
\depends{def:ext-gen,def:L1-loc-p}
Let $(x_t)$ be a PDMP satisfying standard conditions (TODO: write them). Then the domain $\mathcal{D}(\extgen)$ consist of measurable functions from $E$ to $\mathbb{R}$ such that:
\begin{enumerate}
	\item for all $x = (v,\zeta) \in \Gamma$, $\lim_{t \downarrow 0} f(\phi_v(-t,\zeta))$ exists
	\item For all $x = (v,\zeta) \in E$, the function $t\mapsto f(\phi_v(t,\zeta))$ is absolutely continuous on $[0,s_\star(x)[$ with 
	\[
		s_\star(x) = \inf \{t : P_x(T_1 > t) = 0 \}
	\]
	and $T_1$ is the time of the first jump.
	\item Boundary condition: for all $x\in \Gamma$,
	\[
		f(x) = \int_E f(y) Q(dy;x)
	\]
	\item $\mathcal{B} f \in L_1^{loc}(p)$ where
	\[
		\mathcal{B} f(x,s,\omega) = f(x) - f(x_{s^-}(\omega)).
	\]
\end{enumerate}
For $f \in \mathcal{D}(\extgen)$, 
\[
	\extgen f = \VecField f (x) + \lambda(x) \int_E (f(y)-f(x))Q(dy;x)
\]
Where $\VecField f$ can be any representant of the weak derivative of $f$ along the vector field (which are known to exists since $f$ is absolutely continuous along the flow). QUESTION: can we write every function of the equivalent class of $\extgen f$ using the previous formula? (I guess so)
\end{theorem}

\end{document}